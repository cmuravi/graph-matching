\section{Future Directions}
In this paper we proposed several different models for explaining how
recommendation subgraphs arise probabilistically and how optimization 
problems on these graphs can be solved, but there is much more that
can be done.

\begin{enumerate}

\item In practice, there might exist several different
  recommendation subgraphs based on different features. For example,
  two people might be related because they went to the same 
  school, or because they live in the same city, or because they
  would complete a large number of triangles, etc. It's worthwhile
  to investigate how such graphs can be combined into one, or how
  an optimization problem can be solved using all such 
  recommendation subgraphs simultaneously.

\item We should devise metrics that can evaluate how well a
  recommendation subgraph fits a given model and conduct some parameter
  fitting experiments to see how well actual recommendation subgraphs
  which arise in practice fit our models.

\item We should implement the sampling and the greedy algorithms
  given in the paper to see if they can solve to near optimality
  the graph recommendation problems that arise in practice.
\end{enumerate}