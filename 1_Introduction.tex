\begin{abstract}

In this paper, we study the problem of graph recommendations as 
variants of bipartite matching problems. We consider the problem
of solving such matching problems in practice at web-scale. To acheive
this we introduce several models to simulate underlying input graph
structures. We then analyze the conditions under which a random sample 
of edges using constant memory already suffices to be a 
'good' recommendation algorithm as opposed the cases when we may consider
the more classical and involved linear memory polynomial time algorithms. 
We also show how to select the number of recommendations per item while 
building a website so that there exists a 'perfect' graph recommendation.

\end{abstract}

\section{Introduction}

Recommendations are an integral part of several popular websites. Specifically
websites that are highly engaging and typical user behavior involves that
of long browsing sessions almost always contain a recommendation system. For 
e.g., YouTube recommends a set of videos on the right for every video watched
by a user, Quora recommends a set of questions on every question page, Amazon 
recommends similar products on every product page etc. For the rest of this Section,
we will use YouTube as a running example although these observations apply to 
all websites and recommendation systems.

Note that these recommendations are so critical that one could argue that the YouTube
would not be the website it is today if it had no recommendations. A seamless
browse experience that transitions the user from one YouTube video to another directly
dictates the amount of time a user spends on the site and is therefore directly correlated
to both user satisfaction and revenue. 

While recommendations are important, its implementation is performed by finding related
items at runtime. That is, when a user lands on an item the recommendation systems
decide online the set of other items to display. Such systems have the problem that
a global analysis of the performance of the recommendations are hard. For e.g., one
can simply ask if there are an item that was not recommended by even one other item? 

This motivates solving the recommendation problem offline where we can precompute
the set of recommendations for every item. Such a system can be built in two stages. 
A first stage decides on recommendations for each item purely based on relevance and
retrieves $d$ items to show. The second stages prunes it to $c < d$ items such that
globally we optimize to ensure that the resulting recommendation graph is `good'. For 
e.g., we might want to ensure that the resulting graph minimizes the number of items
that was not recommended by a single other video. 

We can represent this notion of recommendations by using a directed graph. A vertex is simply
an item and a directed edge $(u, v)$ is a recommendation from $u$ to $v$. Under this graph
recommendation, for the above problem, if $c=1$ and we require
the graph be strongly connected then our problem reduces to Hamilton cycle, an NP-complete problem.

We can generalize the above graph recommendation problem by using a directed bipartite graph. 
Website owners might not be interested on all the items on their site. We can use one partition $U$
to represent the set of items for which we need recommendations. We can use the other partition $V$
to represent the set of items that are recommended. Note that a single item can be represented in 
both $U$ and $V$ if needed. For our main results we will use this representation. Now the input
to the problem is this directed graph $G$ where each vertex has $d$ recommendations (thanks to the
first stage of the above described backend system). The output that we need is a subgraph $G'$
where each vertex has $c < d$ recommendations. The goal is simply to minimize the number of vertices
that have in-degree less than $a$. Again, note that if $a=c=1$ this is simply the problem of
maximum bipartite matching\cite{}. If $a=1$ and $c > 1$, this is the problem of fractional
matching\cite{}. Both these problems have well-studied polynomial time algorithms\cite{}. For the
rest of the paper, we call this the $(c, a)$-graph recommendation problem.

In practice, the authors have implemented several algorithms in cutting-edge web-technology companies
including Google, Facebook and BloomReach. There are a couple of massive hurdles in the practicality
of backend systems. First is just simplicity. Software and systems need to be simple for it to be
maintainable. It is a nightmare to maintain a complicated system and is avoided almost always
in practice unless absolutely needed. Second is scale. Systems need to scale gracefully and easily. Matching
algorithms require linear memory and super-linear run-time neither of which scales. Facebook graph has
a billion vertices\cite{} and hundreds of billions of edges\cite{}, 
YouTube has XXX videos and YYY recommendations\cite{} etc. Even an ecommerce website with 100M product
pages and 10 recommendations per product would require 100+GB in main memory to run these algorithms. Since
using clusters of single machines with such high memory is prohibitively expensive, map-reduce\cite{} jobs
are needed to solve these problems. Map-reduce for graph problems are firstly notoriously hard and secondly
we are back into building a complicated system. 

The above reasons motivated the authors to study the problem of recommendations as a graph matching
problem and analyze the cases when a simple set of recommendations would suffice and produce near optimal
solutions. Our goal is to find simple relationships between the parameters of the problem and compute
thresholds that dictate the need for the different algorithms. In practice, this would immediately imply 
that a very simple backend system can be built in many cases without the need for a complicated algorithm. 
If the thresholds provided by our analysis are insufficient in quality, 
then the designer can consider implementing the classical algorithms.

