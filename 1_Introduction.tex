\section{Introduction}
As the amount of data a service collects and indexes increases, the
problem of surfacing relevant content arises and may companies face
this challenge today. Netflix needs to recommend similar movies,
Amazon and Bloomreach need to recommend similar products, Facebook
needs to recommend pages or friends. This recommendation problem is
hard to solve and often critical to the success of these services. For
example, when a user signs up for the first time, Facebook needs to
recommend them to as many people relevant as possible very quickly
because if a user does not need reach a certain friend count, his the
probability of retaining that user decreases rapidly. Similarly,
Amazon may want to surface an undersold product on more commonly
visited product pages, but the revenue gain will not be significant
enough unless a certain number of pages recommend the said
product. What makes this problem difficult is that you can only show
so many recommendations in high value spots without overwhelming the
user. So the number of recommendations a high traffic user or product
can make is limited, and these need to be allocated efficiently. This
motivates modeling the problem as a an optimization problem on graphs
as follows. \vs

We let $L$ and $R$ be sets of products or users. We model the
relationship between these sets of products or users by a bipartite
graph $G=(L,R,E)$ where there is an edge between between a $u\in L$
and $v\in R$ if they are related in some way. Our goal is to find
a subgraph $H\subseteq G$ such that $\deg_{H}(u) \leq c$ for all $u\in
L$ and the number of vertices $u\in L$ with $\deg_{G}(u) \geq
a$ is maximized. We will call this problem the $(c,a)$-graph
recommendation problem.\vs

This problem can capture all the examples described above. In the case
of Netflix $L$ is a set of movies and $R$ is a set of users. In the
case of Facebook $L$ is a set of active users and $R$ is a set of new
users who need friends. In Amazon and Bloomreach's case $L$ is a set
of high traffic/revenue products and $R$ is a set of products whose
exposure we seek to increase. \vs

There are special cases of the graph recommendation problem that can
be solved to optimality in polynomial time. In particular if $a=1$
then the problem is simply a maximum cardinality $b$-matching
problem. However, implementing the algorithms that solve $b$-matchings
such as the blossom algorithm in general graphs is tricky as they
involve a good deal of bookkeeping. Furthermore they take $O(v^{2.5})$
time at best and are memory-intensive. With web-scale sizes of the
problems encountered in practice, any solution that requires
superlinear time is unlikely to be practical. In this paper we
develop several linear time algorithms that operate locally and
therefore use very little memory, but still output solutions
that are near optimal in the average-case and worst case.