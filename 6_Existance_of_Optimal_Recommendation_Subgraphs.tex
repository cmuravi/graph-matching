\section{Existance of Optimal Recommendation Subgraphs}
Let $G=(L,R,E)$ be a bipartite graph. We define a \emph{perfect} $(c,a)$-recommendation on $G$ to be a subgraph $H$ such that $deg_H(u)\leq c$ for all $u\in L$ and $deg_H(v)=a$ for all $v\in R$. In this section we will prove a sufficient condition for perfect $(a,a)$-recommendation subgraphs to exist in a bipartite graph where edges are sampled uniformly and independently with probability $p$. Our result, hinges on the following characterization of optimal recommendation subgraphs due to Enomoto, Kano and Ota \cite{EnomotoKanoOta1988}:

\begin{thm}
Let $G=(L,R,E)$ be a bipartite graph, $k=|L|/|R|$ and $\lambda = a - 1 + 1/c$. Then an optimal $(c,a)$-recommendation subgraph exists if 

\begin{itemize}
\item $a \leq ck$
\item for every set $S\subseteq R$ such that $|S| < n/\lambda$, $|N(S)| > \lambda|S|$
\item for every set $S\subseteq R$ such that $|S| \geq n/\lambda$, $N(S) = L$
\end{itemize}
\end{thm}
 
Note that in a perfect $(c,a)$-recommendation subgraph there are exactly $a|R|$ edges. Therefore, the first condition merely states that the number of vertices in $L$ must be high enough to support this many edges given that each of those vertices have degree at most $a$. The other two conditions state that any subset of vertices in $R$ must expand into $L$ by a factor of $\lambda$. The next theorem shows that this expansion property is satisfied with high probability if the edge selection probability is high enough. We will use this result first to prove a special case where $a=c$ and $|L|=|R|=n$. \vs

We note that this case is in fact the problem of finding a spanning $a$-regular subgraph in $G$. There is a wealth of previous work that has been done on generalized matchings and $a$-factors. Pyber has shown that graph of average degree at least $64a^2\log(n)$ contains a not necessarily spanning $a$-regular subgraph and Bollabas et al. noted that this implies that random graphs with the same expected degree have a high probability of containing such subgraphs too \cite{Bollobas2006, Pyber1985}. Furthermore, a stronger and more involved result about the existence of Hamiltonian cycles in graphs with large enough minimum degrees due to Frieze implies this result \cite{Frieze1988}. In this section, we present a short and straightforward analysis in line with the analysis of bipartite expander graphs.

In order to this, we first need to prove a lemma.

\begin{lem}
Suppose that $G=(L,R,E)$ is a bipartite graph with $|L|=|R|=n$ that does not have a perfect $(a,a)$-recommendation subgraph. Then there exists a set $S\subseteq L$ or $S\subseteq R$ satisfying $|S|\leq n/(\lambda+1)$ and $|N(S)| < |S|\lambda$.
\end{lem}

\begin{proof}
First note that when $|L|=|R|$, a perfect $(a,a)$-recommendation subgraph assigns degree $a$ to each vertex in $R$ and degree at most $a$ to each vertex in $L$. In order for there to be the same number edges coming out of $L$ as there are out of $R$, the degree of every vertex such a recommendation subgraph needs to be $a$.\vs

This means that if a $(a,a)$-recommendation subgraph doesn't exists in $G$, then there doesn't exists on in $G$ when we flip the labels of $L$ and $R$. Applying Theorem 11, we now see that there must exists some $S$ that's either a subset of $L$ or $R$ that doesn't expand into a set $T$ that's of size at least $\lambda|S|$. We will assume without loss of generality that $S\subseteq L$ and consider the minimal such $S$ and let $|S|=nb$ for some constant fraction $b$. \vs

However, note that if there are no edges between $S$ and $R\backslash T$, then $N(R\backslash T) = L\backslash S$. Since we picked $S$ to be minimal in size, we must have $|S|\leq |R\backslash T|$ or equivalently, $nb \leq n -nb\lambda$. Rearranging this inequality shows that $b(\lambda+1) \leq 1$ or that $b\leq 1/(\lambda+1)$ as required.
\end{proof}

Using this lemma, we can prove the following theorem:

\begin{thm} 
Let $G=(L,R,E)$ be a bipartite graph with $|L|=|R|=n$ generated randomly by picking $d$ edges out of each vertex $u\in L$. There exists some $c$ that only depends on $a$ such that if $d=c\ln(n)$, then $G$ has a perfect $(a,a)$-recommendation subgraph with probability 1 as $n\to\infty$.
\end{thm}

\begin{proof}
Let $\lambda = a - 1 + 1/a$. Using the lemma above, if $G$ does not have a perfect $(a,a)$-recommendation subgraph, then there must be a set $S$ such that $|S|\leq n/(\lambda+1)$ that's either contained in $R$ or $L$ that expands into a set $T$ of size less than $|S|\lambda$. \vs

We will first deal with the case where $S\subseteq R$. Therefore, we will consider all pairs of sets $S\subseteq R$ such that $|S|<n/(\lambda+1)$ and $T\subseteq L$ such that $|T| = \lambda|S|$. If $N(S)\subseteq T$, then none of the at least $|S|(n-\lambda |S|)$ edges between $S$ and $L\backslash T$ can be present in $G$ (and similarly for $S\subseteq L$). Since $G$ is generated by choosing $d$ neighbors for each vertex $u\in L$, the probability that there is an edge between a vertex $u\in L$ and a vertex $v\in R$ is $p = d/n = c\log(n)/n$. Now, by a union bound, the probability that the desired expansion out of $S$ fails to happen is at most: 

\begin{align*}
       \Pr\left[\bigvee_{\substack{|S|\leq n/(\lambda+1) \\ |T| = \lambda |S|}} \text{$S\subseteq R$ fails to expand out of T}\right]
\leq&  \sum_{s=1}^{n/(\lambda+1)} \binom{n}{s}\binom{n}{s\lambda}(1-p)^{s(n-s\lambda)} \\
\leq&  \sum_{s=1}^{n/(\lambda+1)} \left(\frac{ne}{s}\right)^s \left(\frac{ne}{\lambda s}\right)^{\lambda s} \exp\left(-ps(n-s\lambda)\right) \\
\leq&  \frac{1}{\lambda^\lambda} \sum_{s=1}^{n/(\lambda+1)} \left(\frac{ne}{s}\right)^{s(\lambda+1)} \exp\left(-c\ln(n)s(1-s\lambda/n)\right)
\end{align*}

But note that within the bounds of our sum we have $1-s\lambda/n \geq 1/(\lambda+1)$, so

\[
\Pr\left[\bigvee_{\substack{|S|\leq n/(\lambda+1) \\ |T| = \lambda |S|}} \text{S fails to expand out of T}\right]
\leq \frac{2}{\lambda^\lambda} \sum_{s=1}^{n/(\lambda+1)} \left(\frac{ne}{s}\right)^{s(\lambda+1)} n^{-cs/(\lambda+1)}
\]

We can now take $c=(\lambda+1+\epsilon)^2$ for any $\epsilon>0$ and this sum becomes $o(1)$ as $n\to\infty$. We will now estimate the same probability when $S\subseteq L$. For expansion to not happen, all of the $d|S|$ edges coming out of $S$ need to be contained in some $T\subseteq R$ such that $|T|\leq |S|\lambda$. Each of these edges fall into $T$ with probability $|T|/n = \lambda|S|/n$. Therefore, we can write:

\begin{align*}
       \Pr\left[\bigvee_{\substack{|S|\leq n/(\lambda+1) \\ |T| = \lambda |S|}} \text{$S\subseteq L$ fails to expand out of T}\right]
\leq&  \sum_{s=1}^{n/(\lambda+1)} \binom{n}{s}\binom{n}{s\lambda}\left(\frac{s\lambda}{n}\right)^{ds}\\
\leq&  \sum_{s=1}^{n/(\lambda+1)} \left(\frac{ne}{s}\right)^s \left(\frac{ne}{\lambda s}\right)^{\lambda s} \left(\frac{\lambda}{\lambda+1}\right)^{ds}\\
=&     \sum_{s=1}^{n/(\lambda+1)} \left(\frac{1}{\lambda^\lambda}\left(\frac{ne}{s}\right)^{\lambda+1}\exp(-d(\lambda+1))\right)^s \\
=&     \sum_{s=1}^{n/(\lambda+1)} \left(\frac{1}{\lambda^\lambda}\left(\frac{ne}{s}\right)^{\lambda+1}n^{-c(\lambda+1)}\right)^s \\
\end{align*}

Once again, if we set $c=(\lambda+1+\epsilon)$ for any $\epsilon > 0$, each term of the sum becomes $o(n^{-s\epsilon})$ and the whole some becomes $o(1)$ as required. Since both sums are $o(1)$, the probability that our random graph doesn't have a perfect $(a,a)$-recommendation subgraph tends to 1 as $n\to\infty$.

\end{proof}

Note that with this proof in place, we can note that the result actually holds even when $L$ and $R$ have different sizes. As long as $L$ and $R$ go to $\infty$ together, we could arbitrarily restrict ourselves to an induced subgraph so that both sides of the bipartite graph will be the same size, and wouldn't lose the property that edges are sampled independently and uniformly randomly. The result above would show the existence of a perfect recommendation subgraph with the same parameter $p$. \vs

The case when perfect recommendation subgraphs exist is an interesting special case because if a perfect $(c,a)$-recommendation subgraph exists for a given $G=(L,R,E)$, then it can be found simply in polynomial time even when $c\ne a$. In particular, note that by definition, in perfect $(c,a)$-recommendation subgraph every vertex in $L$ has degree at most $c$ and every vertex in $R$ has degree at most $a$. Therefore, such a subgraph is a feasible solution to the $b$-matching problem with the same constraints. It is in fact a feasible solution with the maximum number of edges since adding any more edges would require us to violate the degree constraint of a vertex in $R$. Therefore, this case can be solved by a standard maximum cardinality $b$-matching algorithm. \vs

We can extend this special case to find optimal $(a,am)$-recommendation graphs for any integer $m$ in random bipartite graphs using a simple partitioning argument. Note that this case is only interesting provided that $|R|\geq |L|/m$, which is the case we constrain ourselves to.

\begin{thm}
Let $G=(L,R,E)$ be a bipartite graph with $|R|\geq |L|/m$ generated randomly by picking each edge independently and uniformly with probability $p$. There exists some $c$ that only depends on $a$ such that if $p=c\ln(n)/n$, then $G$ has an optimal $(a,am)$-recommendation subgraph with probability 1 as $|L|,|R|\to\infty$.
\end{thm}
\begin{proof}
Let $|R|=r$ and $|L|=l$. Note that given the degree constraints on $L$, the maximum number of vertices that can have degree $\geq am$ is at most $l/m$. Therefore, we can restrict $R$ to subset of itself $R'$ of size $l/m$. Next, we partition $L$ into $m$ subsets $L_1,\ldots, L_m$ of size $l/m$ each. Note that the induced subgraphs $G[L_1,R'], \ldots, G[L_m,R']$ are all generated by picking each edge independently and uniformly with probability $p$. For each one, there exists a setting of $c$ such that if $p=c\ln(n)/n$, then each one of these graphs fail to contain an perfect $(a,a)$-recommendation subgraph with probability $o(1)$. Therefore, assuming that $m$ is a constant, we can assert by a union bound that each of these graphs contains a perfect $(a,a)$-recommendation subgraph. Superimposing these subgraph in $G[L,R']$ now gives us a subgraph $H$ such that $\deg_H(u) = a$ for all $u\in L$ and $\deg_H(v)=ma$ for all $v$. This proves the claim once we let $|L|$ and $|R|$ scale together towards $\infty$.
\end{proof}

Note that if an optimal $(a,am)$-recommendation subgraph exists, we can still recover it using a $b$-matchings as above. We simply need to arbitrarily restrict $R$ to an appropriately sized subset of itself and run the $b$-matching solution we had on the $m$ graphs given by $G[L_1,R'], \ldots, G[L_m,R']$.
